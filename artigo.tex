%%%%%%%%%%%%%%%%%%%%%%%%%%%%%%%%%%%%%%%%%%%%%%%%%%%%%%%%%%%%%%%%%%%%%%%%%%%%%%%%
% artigo.tex
%
% Modelo para TCC em formato de artigo
%
% Para maiores informações, visite:
%    https://github.com/abrantesasf/latex
%
% ALTERE DE ACORDO COM SUA NECESSIDADE.
%%%%%%%%%%%%%%%%%%%%%%%%%%%%%%%%%%%%%%%%%%%%%%%%%%%%%%%%%%%%%%%%%%%%%%%%%%%%%%%%


%%%%%%%%%%%%%%%%%%%%%%%%%%%%%%%%%%%%%%%%%%%%%%%%%%%%%%%%%%%%%%%%%%%%%%%%%%%%%%%%
%%% Classe do documento
\documentclass[12pt]{article}

%%%%%%%%%%%%%%%%%%%%%%%%%%%%%%%%%%%%%%%%%%%%%%%%%%%%%%%%%%%%%%%%%%%%%%%%%%%%%%%%
%%% Preâmbulo com todas as outras outras chamadas para todos os outros packages
%%% e o que mais for necessário
\input{utils/preambulo.tex}

%%%%%%%%%%%%%%%%%%%%%%%%%%%%%%%%%%%%%%%%%%%%%%%%%%%%%%%%%%%%%%%%%%%%%%%%%%%%%%%%
%%% Ajuste do layout, espaçamento de linhas e etc.:
\geometry{a4paper, portrait,
  left=2cm, right=2cm,
  top=2cm, bottom=2cm}
%\onehalfspacing

%%%%%%%%%%%%%%%%%%%%%%%%%%%%%%%%%%%%%%%%%%%%%%%%%%%%%%%%%%%%%%%%%%%%%%%%%%%%%%%%
%%% Configurações para as propriedades do PDF:
\hypersetup{
   %hidelinks,           % Comente para web, descomente para impressão
   colorlinks=true,      % True para web, False para impressão
   pdftitle={Título do Artigo},
   pdfauthor={Autor(es)},
   pdfsubject={Assunto principal do artigo},
   pdfkeywords={palavras-chave para o PDF},
   pdfinfo={
      CreationDate={}, % Ex.: D:AAAAMMDDHH24MISS
      ModDate={}       % Ex.: D:AAAAMMDDHH24MISS
   }
}

%%%%%%%%%%%%%%%%%%%%%%%%%%%%%%%%%%%%%%%%%%%%%%%%%%%%%%%%%%%%%%%%%%%%%%%%%%%%%%%%
%%% Compilação condicional de seções
%\includeonly{}

%%%%%%%%%%%%%%%%%%%%%%%%%%%%%%%%%%%%%%%%%%%%%%%%%%%%%%%%%%%%%%%%%%%%%%%%%%%%%%%%
%%% Começa o documento
\begin{document}

%%%%%%%%%%%%%%%%%%%%%%%%%%%%%%%%%%%%%%%%%%%%%%%%%%%%%%%%%%%%%%%%%%%%%%%%%%%%%%%%
%%% Front matter

% Bookmark para a capa do artigo
\pdfbookmark[1]{Início}{titulo}

% Título (se necessário um subtítulo, usar quebra de linha e o tamanho large):
\title{\textbf{Título do TCC}}
%\bookmark[dest=titulo,level=chapter]{Title Page}

% Autor: comente a linha do segundo autor se apenas 1 único
\author{Primeiro Autor}
\author{Segundo Autor}
\affil{Universidade Vila Velha}

% Data
\date{202x-xx-xx}

% Gera os títulos:
\maketitle

% Resumo em português:
\abstract{Lorem ipsum dolor sit amet, consectetur adipiscing elit. Aenean ornare
  turpis leo, id mollis risus dignissim id. Aliquam vel consequat
  lacus. Suspendisse malesuada varius ipsum, vitae fringilla nulla fringilla
  ac. Ut a sapien a tellus laoreet tincidunt bibendum id diam. Mauris porta
  risus et nisi eleifend rutrum. Duis ut volutpat risus. Nulla lacus ante,
  tincidunt sit amet libero id, porttitor auctor quam. Ut nisl tortor, blandit
  eget odio nec, ornare aliquet turpis. Etiam sit amet fermentum justo. Aliquam
  a dolor libero. Praesent malesuada, sapien in efficitur pharetra, odio lectus
  tincidunt tortor, scelerisque hendrerit mauris leo sed tellus. Cras ut neque
  at velit interdum dignissim.}

% Sumário:
\pdfbookmark[1]{Sumário}{sumario}
\tableofcontents

%%%%%%%%%%%%%%%%%%%%%%%%%%%%%%%%%%%%%%%%%%%%%%%%%%%%%%%%%%%%%%%%%%%%%%%%%%%%%%%%
%%% Main matter
\newpage
%%%%%%%%%%%%%%%%%%%%%%%%%%%%%%%%%%%%%%%%%%%%%%%%%%%%%%%%%%%%%%%%%%%%%%%%%%%%%%%%
\section{Introdução}
\label{sec:intro}

Apresentação clara do tema que está sendo trabalhado e do problema cuja proposta
de solução será desenvolvida. Deve-se explicar a atualidade do tema e do
problema, bem como a importância no meio social e acadêmico. Se o conhecimento
sobre o problema apresentar lacunas, aponte.

Faça uma apresentação objetiva, com as próprias palavras, para demonstrar que
você estudou e compreendeu a complexidade do problema, mesmo que essa
apresentação não seja tão profunda e detalhada (isso será desenvolvido ao longo
do TCC).

Lembre-se: para escrever \textbf{qualquer parte} do TCC é necessário que o
estudo e a revisão bibliográfica já tenham sido executados (ou que estejam em
andamento). Você só consegue escrever bem se estudou muito.

Você deve demonstrar que conseguiu escolher um tema de trabalho, que conseguiu
delimitar o tema para que o TCC não fique muito abrangente e que conseguiu
escolher um problema específico para trabalho. Demonstre que você delimitou com
clareza e entende perfeitamente os limites do problema que você está tentando
resolver.

Após demonstrar a escolha e a delimitação do tema, disserte sobre o foco do
problema contextualizando-o em relação ao segmento, à empresa, à sociedade, e/ou
à época. Apresentar o objeto de pesquisa.

Esta seção deve ter aproximadamente 2--3 páginas.


%%%%%%%%%%%%%%%%%%%%%%%%%%%%%%%%%%%%%%%%%%%%%%%%%%%%%%%%%%%%%%%%%%%%%%%%%%%%%%%%
%%% Apêndices
%\appendix
%%%%%%%%%%%%%%%%%%%%%%%%%%%%%%%%%%%%%%%%%%%%%%%%%%%%%%%%%%%%%%%%%%%%%%%%%%%%%%%%%
% placeholder.tex
%
% Modelo de arquivo para uso com a classe uvvTeX2, para a formatação de
% trabalhos acadêmicos na Universidade Vila Velha (UVV) (https://www.uvv.br).
%
% Para maiores informações, visite:
%    https://github.com/uvv-computacao/uvvtex2
%
% Este placedholder apenas indica uma forma comum de incluir imagens em seu
% trabalho. Use o código a seguir como base e consulte a documentação do
% LaTeX para maiores informações.
%%%%%%%%%%%%%%%%%%%%%%%%%%%%%%%%%%%%%%%%%%%%%%%%%%%%%%%%%%%%%%%%%%%%%%%%%%%%%%%%

\begin{figure}[htbp]
\centering
\caption{Caption da figura}
\label{fig:xxx}
  \fbox{
     \includegraphics[scale=1.0]{imagens/arquivo.png}
  }
  \legend{Legenda}
\end{figure}


%%%%%%%%%%%%%%%%%%%%%%%%%%%%%%%%%%%%%%%%%%%%%%%%%%%%%%%%%%%%%%%%%%%%%%%%%%%%%%%%
%%% Back matter
%\bibliography{utils/biblioteca}
%\printindex

%%%%%%%%%%%%%%%%%%%%%%%%%%%%%%%%%%%%%%%%%%%%%%%%%%%%%%%%%%%%%%%%%%%%%%%%%%%%%%%%
%%% Termina o documento
\end{document}

