%%%%%%%%%%%%%%%%%%%%%%%%%%%%%%%%%%%%%%%%%%%%%%%%%%%%%%%%%%%%%%%%%%%%%%%%%%%%%%%%
% referencial.tex
%
% Modelo de arquivo para uso com a classe uvvTeX2, para a formatação de
% trabalhos acadêmicos na Universidade Vila Velha (https://uvv.br).
%
% Para maiores informações, visite:
%    https://github.com/uvv-computacao/uvvtex2
%
% Este modelo mostra como inserir o capítulo de introdução de sua monografia.
% Basta informar o título e o label da introdução, e escrever o conteúdo.
%%%%%%%%%%%%%%%%%%%%%%%%%%%%%%%%%%%%%%%%%%%%%%%%%%%%%%%%%%%%%%%%%%%%%%%%%%%%%%%%


%%%%%%%%%%%%%%%%%%%%%%%%%%%%%%%%%%%%%%%%%%%%%%%%%%%%%%%%%%%%%%%%%%%%%%%%%%%%%%%%
\chapter{Revisão bibliográfica}
\label{sec:revbib}

Neste capítulo você deve fazer um levantamento bibliográfico detalhado e
abrangente. Deve buscas trabalhos semelhantes e deve demonstrar que você estudo
``tudo'' que é necessário para o desenvolvimento de seu trabalho.

Não há uma forma padrão de escrever a revisão bibliográfica, você pode dividir
este capítulo em seções e subseções da maneira que for mais conveniente para
você. O importante é que você demonstre o estado atual do conhecimento e que
demonstre que você entendeu do assunto o suficiente para ser capaz de
desenvolver o trabalho e alcançar os objetivos.

É importante que durante a revisão bibliográfica e, na verdade, durante todo seu
TCC, você cite adequadamente as fontes da revisão bibliográfica
\cite{Thompson1998}.

Segundo \citeauthoronline{Graham1994} (\citeyear{Graham1994}), siga as normas da
ABNT para fazer as citações. Utilize programas como o
\href{https://www.jabref.org/}{JabRef} para manter suas referências.

