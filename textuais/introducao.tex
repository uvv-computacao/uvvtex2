%%%%%%%%%%%%%%%%%%%%%%%%%%%%%%%%%%%%%%%%%%%%%%%%%%%%%%%%%%%%%%%%%%%%%%%%%%%%%%%%
% introducao.tex
%
% Modelo de arquivo para uso com a classe uvvTeX2, para a formatação de
% trabalhos acadêmicos na Universidade Vila Velha (https://uvv.br).
%
% Para maiores informações, visite:
%    https://github.com/uvv-computacao/uvvtex2
%
% Este modelo mostra como inserir o capítulo de introdução de sua monografia.
% Basta informar o título e o label da introdução, e escrever o conteúdo.
%%%%%%%%%%%%%%%%%%%%%%%%%%%%%%%%%%%%%%%%%%%%%%%%%%%%%%%%%%%%%%%%%%%%%%%%%%%%%%%%


%%%%%%%%%%%%%%%%%%%%%%%%%%%%%%%%%%%%%%%%%%%%%%%%%%%%%%%%%%%%%%%%%%%%%%%%%%%%%%%%
\chapter{Introdução}
\label{sec:intro}


%%%%%%%%%%%%%%%%%%%%%%%%%%%%%%%%%%%%%%%%
\section{O problema}
\label{sec:intro:prob}

Apresentação clara do tema que está sendo trabalhado e do problema cuja proposta
de solução será desenvolvida no TCC. Deve-se explicar a atualidade do tema e do
problema, bem como a importância no meio social e acadêmico. Se o conhecimento
sobre o problema apresentar lacunas, aponte.

Faça uma apresentação objetiva, com as próprias palavras, para demonstrar que
você estudou e compreendeu a complexidade do problema, mesmo que essa
apresentação não seja tão profunda e detalhada (isso será desenvolvido ao longo
do TCC).

Lembre-se: para escrever \textbf{qualquer parte} do TCC é necessário que o
estudo e a revisão bibliográfica já tenham sido executados (ou que estejam em
andamento). Você só consegue escrever bem se estudou muito.

Você deve demonstrar que conseguiu escolher um tema de trabalho, que conseguiu
delimitar o tema para que o TCC não fique muito abrangente e que conseguiu
escolher um problema específico para trabalho. Demonstre que você delimitou com
clareza e entende perfeitamente os limites do problema que você está tentando
resolver.

Após demonstrar a escolha e a delimitação do tema, disserte sobre o foco do
problema contextualizando-o em relação ao segmento, à empresa, à sociedade, e/ou
à época. Apresentar o objeto de pesquisa.


%%%%%%%%%%%%%%%%%%%%%%%%%%%%%%%%%%%%%%%%
\section{Formulação do problema}
\label{sec:intro:form_prob}

A formulação do problema nada mais é do que a proposição de uma questão que se
buscará responder por meio de pesquisa. Em outras palavras, problema é a
pergunta que a pesquisa pretende resolver. O problema deve ser:

\begin{itemize}
\item Claro e preciso
\item Empírico
\item Delimitado
\item Passível de solução
\item Sem natureza valorativa (é bom, é certo, etc.)
\end{itemize}

Escreva o problema como uma \textbf{pergunta}. É essa pergunta que você vai
tentar solucionar no TCC. Escreva uma única frase que seja capaz de esclarecer
ao leitor exatamente qual é o problema que você está tentando solucionar.


%%%%%%%%%%%%%%%%%%%%%%%%%%%%%%%%%%%%%%%%
\section{Hipótese}
\label{sec:intro:hip}

No contexto do TCC a hipótese pode ser entendida como sendo a solução provisória
ou tentada para um determinado problema. Ela designa a proposição a partir da
qual o TCC busca resolver o problema, mostra como procuramos resolver o problema
formulado.

Deve ser uma sentença afirmativa que se faz em forma declarativa. Deve ser uma
frase clara, sem conter julgamentos morais, específica ao problema formulado,
plausível e realizável dentro do tempo de desenvolvimento do TCC.

Não deve ser escrita de forma interrogativa ou condicional.


%%%%%%%%%%%%%%%%%%%%%%%%%%%%%%%%%%%%%%%%
\section{Objetivos}
\label{sec:intro:obj}

Os objetivos constituem a finalidade de um trabalho científico, ou seja, a meta
que se pretende atingir com a elaboração da pesquisa. São eles que indicam o que
um pesquisador realmente deseja fazer. Sua definição clara ajuda em muito na
tomada de decisões quanto aos aspectos metodológicos da pesquisa, afinal, temos
que saber o que queremos fazer, para depois resolvermos como proceder para
chegar aos resultados pretendidos.

A formulação dos objetivos --- seja o objetivo geral, seja dos específicos ---
se faz mediante o emprego de verbos no infinitivo: contribuir, analisar,
descrever, investigar, comparar, etc.

Note que os objetivos têm função norteadora no momento da leitura e avaliação do
TCC ou da tese. Isto porque, um trabalho acadêmico é julgado, em grande parte,
pela capacidade de cumprir os objetivos que se propõem em suas páginas
iniciais. Então, o alerta é: cuidado na hora de estabelecer os objetivos. Além
de claros, estes têm que ser possíveis de serem realizados.

\subsection{Objetivo geral}
\label{sec:intro:obj:ger}

O que o autor deseja descobrir/realizar/criar sobre o problema escolhido e a
hipótese de solução. Representa a ação global que o levará à solução do
problema. Deve ser sempre expresso de forma geral.

Como o próprio nome diz, o objetivo geral é aquele mais amplo. É a meta de longo
alcance, a contribuição que se deseja oferecer com a execução do TCC. Em geral,
o primeiro e maior objetivo do pesquisador é o de obter uma resposta
satisfatória ao seu problema de pesquisa.

Para a definição do objetivo geral, é recomendado o uso de verbos com
significado abrangente. Deve englobar a totalidade do problema, definindo de
forma clara o que se pretende no final do projeto.

Verbos para objetivos gerais que se focam em conceitos:
\begin{itemize}
\item conhecer;
\item compreender;
\item entender;
\item identificar;
\item reconhecer;
\item generalizar.
\end{itemize}

Verbos para objetivos gerais que se focam em procedimentos:
\begin{itemize}
\item desenvolver;
\item estabelecer;
\item organizar;
\item capacitar;
\item demonstrar.
\end{itemize}

Verbos para objetivos gerais que se focam em atitudes:
\begin{itemize}
\item contribuir;
\item colaborar;
\item valorizar;
\item interiorizar;
\item mostrar.
\end{itemize}

\subsection{Objetivos específicos}
\label{sec:intro:obj:esp}

Para se cumprir os objetivos gerais é preciso delimitar metas mais específicas
dentro do trabalho. São elas que, somadas, conduzirão ao desfecho do objetivo
geral.

A lista dos objetivos intermediários, das etapas a serem concluídas para se
alcançar o objetivo geral. Os objetivos específicos são indispensáveis para a
definição da metodologia e demais recursos a serem utilizados.

Para a definição de objetivos específicos, é recomendado o uso de verbos
com significado mais restrito e direcionado. Os objetivos específicos
contribuem para a concretização do objetivo geral, pormenorizando-o. Estão
relacionados com as áreas específicas nas quais se desenvolvem.

Verbos usados em objetivos específicos para indicar análise:
\begin{itemize}
\item analisar;
\item investigar;
\item comprovar;
\item classificar;
\item comparar;
\item contrastar;
\item diferenciar;
\item distinguir.
\end{itemize}

Verbos usados em objetivos específicos para indicar avaliação:
\begin{itemize}
\item avaliar;
\item pesquisar;
\item selecionar;
\item precisar;
\item decidir;
\item estimar;
\item medir;
\item validar.
\end{itemize}

Verbos usados em objetivos específicos para indicar compreensão:
\begin{itemize}
\item concluir;
\item inferir;
\item deduzir;
\item interpretar;
\item determinar;
\item descrever;
\item ilustrar.
\end{itemize}

Verbos usados em objetivos específicos para indicar conhecimento:
\begin{itemize}
\item registrar;
\item definir;
\item identificar;
\item nomear;
\item especificar;
\item exemplificar;
\item enumerar;
\item citar.
\end{itemize}

Verbos usados em objetivos específicos para indicar síntese:
\begin{itemize}
\item esquematizar;
\item organizar;
\item constituir;
\item estruturar;
\item generalizar;
\item documentar;
\item desenvolver.
\end{itemize}

Verbos usados em objetivos específicos para indicar aplicação:
\begin{itemize}
\item aplicar;
\item praticar;
\item empregar;
\item operar;
\item usar.
\end{itemize}


%%%%%%%%%%%%%%%%%%%%%%%%%%%%%%%%%%%%%%%%
\section{Justificativa}
\label{sec:intro:jus}

Aqui você deve identificar a importância da escolha do tema para o segmento, o
setor, a sociedade e sendo um estudo de caso, para o campo a ser
pesquisado. Pense na justificativa como sendo uma apresentação rápida, de cinco
minutos, do seu projeto para um possível financiador de seu projeto ou comprador
de sua idéia ou produto: você deve demonstrar, em poucas linhas, que o projeto é
importante.
