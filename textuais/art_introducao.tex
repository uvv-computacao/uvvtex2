%%%%%%%%%%%%%%%%%%%%%%%%%%%%%%%%%%%%%%%%%%%%%%%%%%%%%%%%%%%%%%%%%%%%%%%%%%%%%%%%
\section{Introdução}
\label{sec:intro}

Apresentação clara do tema que está sendo trabalhado e do problema cuja proposta
de solução será desenvolvida. Deve-se explicar a atualidade do tema e do
problema, bem como a importância no meio social e acadêmico. Se o conhecimento
sobre o problema apresentar lacunas, aponte.

Faça uma apresentação objetiva, com as próprias palavras, para demonstrar que
você estudou e compreendeu a complexidade do problema, mesmo que essa
apresentação não seja tão profunda e detalhada (isso será desenvolvido ao longo
do TCC).

Lembre-se: para escrever \textbf{qualquer parte} do TCC é necessário que o
estudo e a revisão bibliográfica já tenham sido executados (ou que estejam em
andamento). Você só consegue escrever bem se estudou muito.

Você deve demonstrar que conseguiu escolher um tema de trabalho, que conseguiu
delimitar o tema para que o TCC não fique muito abrangente e que conseguiu
escolher um problema específico para trabalho. Demonstre que você delimitou com
clareza e entende perfeitamente os limites do problema que você está tentando
resolver.

Após demonstrar a escolha e a delimitação do tema, disserte sobre o foco do
problema contextualizando-o em relação ao segmento, à empresa, à sociedade, e/ou
à época. Apresentar o objeto de pesquisa.

Esta seção deve ter aproximadamente 2--3 páginas.
