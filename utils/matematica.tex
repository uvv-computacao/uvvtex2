%%%%%%%%%%%%%%%%%%%%%%%%%%%%%%%%%%%%%%%%%%%%%%%%%%%%%%%%%%%%%%%%%%%%%%%%%%%%%%%%
% matematica.tex
%
% Arquivo de configuração de packages para uso com a classe uvvTeX2, para a
% formatação de trabalhos acadêmicos na Universidade Vila Velha (UVV)
% (https://www.uvv.br).
%
% Para maiores informações, visite:
%    https://github.com/uvv-computacao/uvvtex2
%
% Geralmente não é necessário alterar nada aqui!
%%%%%%%%%%%%%%%%%%%%%%%%%%%%%%%%%%%%%%%%%%%%%%%%%%%%%%%%%%%%%%%%%%%%%%%%%%%%%%%%

% Carrega bibliotecas e símbolos matemáticos, fontes adicionais e configura
% algumas outras opções
\usepackage{amsmath}
\usepackage{amssymb}
\usepackage{amsthm}
\usepackage{amsfonts}
\usepackage{mathrsfs}
\usepackage{proof}
\usepackage{siunitx}
  \sisetup{group-separator = {\,}}
  \sisetup{group-digits = {integer}}
  \sisetup{output-decimal-marker = {,}}
  \sisetup{separate-uncertainty}
  \sisetup{multi-part-units = single}
  \sisetup{binary-units = true}
  \sisetup{list-final-separator = { e }}
\usepackage{syllogism}
  \setsylpuncpa{}
  \setsylpuncpb{}
  \setsylpuncc{}
  \setsylergosign{}
\usepackage{bm}
\usepackage{cancel}
\usepackage{esvect}
\usepackage{mathtools}
\usepackage{icomma}
\usepackage{nicefrac}

% Altera separador decimal via comando, se necessário (prefira o siunitx):
%\mathchardef\period=\mathcode`.
%\DeclareMathSymbol{.}{\mathord}{letters}{"3B}

% Declara alguams unidades adicionais para siunitx:
\DeclareSIUnit{\degreeFahrenheit}{\unit{\degree}F}
\DeclareSIUnit{\gravity}{\mbox{$g$}}



% Definições para teoremas, etc.
\makeatletter
\@ifclassloaded{article}{
  \theoremstyle{definition}
  \newtheorem{definicao}{Definição}[section]
  \newtheorem{conjecture}{Conjectura}[section]
  \newtheorem{teorema}{Teorema}[section]
  \newtheorem{lemma}{Lema}[section]
  \newtheorem{corolario}{Corolário}[section]
}{}
\makeatother

\makeatletter
\@ifclassloaded{book}{
  \theoremstyle{definition}
  \newtheorem{definicao}{Definição}[chapter]
  \newtheorem{conjecture}{Conjectura}[chapter]
  \newtheorem{teorema}{Teorema}[chapter]
  \newtheorem{lemma}{Lema}[chapter]
  \newtheorem{corolario}{Corolário}[chapter]
}{}
\makeatother

\makeatletter
\@ifclassloaded{uvvtex2}{
  \theoremstyle{definition}
  \newtheorem{definicao}{Definição}[chapter]
  \newtheorem{conjecture}{Conjectura}[chapter]
  \newtheorem{teorema}{Teorema}[chapter]
  \newtheorem{lemma}{Lema}[chapter]
  \newtheorem{corolario}{Corolário}[chapter]
}{}
\makeatother
